
We consider the problem of learning from sparse and underspecified
rewards, where an agent receives a complex input, such as a natural
language instruction, and needs to generate a complex response, such
as an action sequence, while only receiving binary success-failure
feedback.  Such success-failure rewards are often underspecified: they
do not distinguish between purposeful and accidental
success. Generalization from underspecified rewards hinges on
discounting spurious trajectories that attain accidental success,
while learning from sparse feedback requires effective exploration.
We address exploration by using a mode covering direction of KL
divergence to collect a diverse set of successful trajectories,
followed by a mode seeking KL divergence to train a robust policy.  We
propose Meta Reward Learning (MeRL) to construct an auxiliary reward
function that provides more refined feedback for learning.  The
parameters of the auxiliary reward function are optimized with respect
to the validation performance of a trained policy. The MeRL approach
outperforms an alternative method for reward learning based on
Bayesian Optimization, and achieves the state-of-the-art on
weakly-supervised semantic parsing. It improves previous work by 1.2\%
and 2.4\% on \textsc{WikiTableQuestions} and \textsc{WikiSQL} datasets
respectively.
